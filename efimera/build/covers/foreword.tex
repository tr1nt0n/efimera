\documentclass[11pt]{article}
\usepackage{fontspec}
\usepackage[utf8]{inputenc}
\setmainfont{Bodoni 72 Book}
\usepackage[paperwidth=17in,paperheight=11in,margin=1in,headheight=0.0in,footskip=0.5in,includehead,includefoot,portrait]{geometry}
\usepackage[absolute]{textpos}
\TPGrid[0.5in, 0.25in]{23}{24}
\parindent=0pt
\parskip=12pt
\usepackage{nopageno}
\usepackage{graphicx}
\graphicspath{ {./images/} }
\usepackage{amsmath}
\usepackage{tikz}
\newcommand*\circled[1]{\tikz[baseline=(char.base)]{
            \node[shape=circle,draw,inner sep=1pt] (char) {#1};}}

\begin{document}

\vspace*{5\baselineskip}

\begingroup
\begin{center}
\huge FOREWORD
\end{center}
\endgroup

\vspace*{2\baselineskip}

\begingroup
\begin{center}
``At times I feel as if I had lived all this before and that I have already written these very words, but I know it was not I: it was another woman, who kept her notebooks so that one day I could use them. I write, she wrote, that memory is fragile and the space of a single life is brief, passing so quickly that we never get a chance to see the relationship between events; we cannot gauge the consequences of our acts, and we believe in the fiction of past, present, and future, but it may also be true that everything happens simultaneously." \\ \textbf{Isabel Allende ( English approximation )}
\end{center}
\endgroup

\begingroup
\begin{center}
``Some things you forget. Other things you never do. But it's not. Places, places are still there. If a house burns down, it's gone, but the place - the picture of it - stays, and not just in my memory, but out there, in the world. What I remember is a picture floating around out there outside my head. I mean, even if I don't think it, even if I die, the picture of what I did, or knew, or saw is still out there. Right in the place where it happened." \\ \textbf{Toni Morrison}
\end{center}
\endgroup

\begingroup
\begin{center}
``. . . wherever they might be they always remember that the past was a lie, that memory has no return, that no spring past could ever be recaptured, and that the most untamed and steadfast love was an ephemeral truth in the end.'' \\ \textbf{Gabriel García Márquez ( English approximation )}
\end{center}
\endgroup


\end{document}